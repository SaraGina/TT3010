\documentclass{article}
\usepackage[utf8]{inputenc}

\title{TT3010 - Audio technology and room acoustics. \newline Exercise 8 - Music instruments.}


%\author{Jan Arne Bosnes}
\date{\today}

\usepackage{natbib}
\usepackage{graphicx}
\usepackage{multicol}
\usepackage{gensymb}
\usepackage{float}
\usepackage{wasysym}


\begin{document}

\maketitle

All tasks are based on chapter 10 in Rossings "Science of Sound" \cite{rossing}. 
It is recommended that the student will try to do every task, but tasks marked \textit{Mandatory} is to be handed in for approval (online). Deadline is November 13.

\section*{Tasks}
%\begin{multicols}{2}
\begin{itemize}
    
    \item[1.] Calculate the frequencies of the first three resonances of closed tubes having length of 275 and 375 cm. Compare these to the resonances shown in the impedance curves of the French horn trombone (figure 11.12).
    
    \item[2.] Determine the frequencies of the trumpet resonances as accurately as you can from fig. 11.7. How closely do they correspond to the bugle notes: B$^\flat_3$, F$^\flat_4$, B$^\flat_4$, D$_5$, F$_5$, A$^\flat_5$ and B$^\flat_5$ (written C$_4$, G$_4$, E$_5$, G$_5$, B$^\flat$ and C$_6$)?

    \item[3.] Pressing the first valve of a trumpet or tuba increases the acoustic length by 12.2 \% and lowers the pitch a whole tone. How long must the first valve slide be in each of these instruments? (In other words, how many cm of tubing should be added to produce the pitch change?) If possible, compare your answers to the measured lengths in actual instruments.
    
    \item[4.] Find the frequency ratios of the first five peaks in fig. 11.4 to the corresponding peaks in fig. 11.6.
    
    
    

\end{itemize}

%\end{multicols}

\end{document}

