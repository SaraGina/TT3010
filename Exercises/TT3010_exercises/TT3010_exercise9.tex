\documentclass{article}
\usepackage[utf8]{inputenc}

\title{TT3010 - Audio technology and room acoustics. \newline Exercise 9 - Music instruments.}


%\author{Jan Arne Bosnes}
\date{\today}

\usepackage{natbib}
\usepackage{graphicx}
\usepackage{multicol}
\usepackage{gensymb}
\usepackage{float}
\usepackage{wasysym}


\begin{document}

\maketitle

All tasks are based on chapter 12 in Rossings "Science of Sound" \cite{rossing}. 
It is recommended that the student will try to do every task, but tasks this does not matter for the access at the exam (already decide by the other exercises)

\section*{Tasks}
%\begin{multicols}{2}
\begin{itemize}
   
   
    \item[1.] Is it possible to fit a flute-type head joint to a clarinet, so that it plays in the manner of a flute. What would you expect the lowest note to be?
    
    \item[2.] 
        \begin{itemize}
            \item[a.] making use of the fact that the speed of an air jet is approximately proportional  to the square root of the blowing pressure, show that doubling the blowing pressure increases the jet speed by 40 \%.
            \item[b.] How much does the jet speed increase when the blowing pressure is tripled?
        \end{itemize}

    \item[3.] Show that the frequencies of the notes in the different  clarinet that are fingered nearly the same are in the ratio of 1 : 3 : 5 (see. fig. 12.10; the frequencies of the notes can be found in table 9.2).
    

    \item[4.] The length of a flute (from the embouchure hole to  the open end) is about 60 cm. What do you expect the frequency of the lowest note  to be?
    

\end{itemize}

%\end{multicols}

\end{document}

